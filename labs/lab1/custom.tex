% The format (A5) is selected to facilitate reading on small
% devices and should NOT be changed.
\documentclass[a5paper,10pt,oneside]{article}

% The package babel is loaded for Swdish with Swedish hyphenation,
% replaces "Contents" with "Innehållsförteckning, "References"
% with "Litteraturförteckning", etc.
\usepackage[swedish]{babel}

\usepackage[T1]{fontenc}

% The package "inputenc" lets us specify what character encoding
% has been used to save the .tex file. If your computer runs
% Linux, the encoding is probably "utf8" by default, while under
% Windows the default will probably be "latin1" The wrong
% character encoding may give strange signs instead of "å", "ä"
% and "ö" or may result in compilation errors.

%\usepackage[latin1]{inputenc} % Probably right if you use Windows
\usepackage[utf8]{inputenc}  % Probably right if you use Linux

% The packages listed below are optional and can be removed if you
% don't use them
\usepackage{graphicx}
\usepackage{cite}
\usepackage{url}
\usepackage{ifthen}
\usepackage{listings}
\usepackage{dirtree}

% These two lines set up options for the listings package and
% can be removed if you don't use it, or changed if you, e.g,
% use another language than Java.
% For more information about the listings package see:
% ftp://ftp.tex.ac.uk/tex-archive/macros/latex/contrib/listings/listings.pdf
\def \lstlistingname {Kodexempel}
\lstset{language=Java,tabsize=3,numbers=left,frame=L,floatplacement=hbtp}

\usepackage{ifpdf}
\ifpdf
	\usepackage[hidelinks]{hyperref}
\else
	\usepackage{url}
\fi

% Change NR and TITLE below to appropriate values

\title{Lab 1 Report \\ Digital Forensics}

% Write the name and user namn for all participants in the group here.
% Separate persons with \and
\author{Anton Fluch \url{anfl4215} \and Johan Bäckström \url{jobc5829}}

\begin{document}

\maketitle

% Here the actual report starts. Everything from here to the start of the
% bibilography should, of course, be removed before you start writing your
% own text.

% INTRODUCTION
\section{Introduction}
The evidence for the case where provided in a .zip file named Lab1.zip. This file produced the following hash sums:
\\
\\ SHA256
\\ {\footnotesize 9c5d0bfbeccd75858426cfc84345e0a68687b0fc5662b715153aa88cefd60fba}
\\
\\ MD5
\\ {\footnotesize c4a731672747131b8b457a77178ad386}

\newpage
When opening the zip file the following folders and files where present: \\
\dirtree{%
.1 Lab1.
.2 Exercise1\textunderscore Hashing.
.3 erase.
.3 erase.exe.
.3 hello.
.3 hello (2).
.3 hello (3).
.3 hello (4).
.3 hello.exe.
.2 Exercise2\textunderscore File\textunderscore Identification.
.3 01.
.3 02.
.3 03.
.3 04.
.3 05.
.3 06.
.3 07.
.3 08.
.3 09.
.3 10.
.3 11.
.3 12.
.2 Exercise3\textunderscore Anti\textunderscore Files\textunderscore Forensics.
.3 c.mp3.
.3 Suspicious\textunderscore File.
.2 Exercise4\textunderscore Acquisition.
.3 winxp.dvi.
.2 Exercise5\textunderscore Cracking.
.3 casssh.pdf.
.3 ht.zip.tar.gpg.
.3 Untitled 1.ods.
.3 untitled.docx.
.3 untitled\textunderscore hash.txt.
.3 wallet1.dat.
.3 wallet2.
.2 Exercise6\textunderscore Steganography.
.3 c1l.png.
.3 c2l.png.
}
\section{Exercise 1 - Hashing}
\section{Exercise 2 - Hashing}
\section{Exercise 3 - Hashing}
\section{Exercise 4 - Hashing}
\section{Exercise 5 - Hashing}
\section{Exercise 6 - Hashing}

%\bibliographystyle{plain}
%\bibliography{bibtex}
%\bibdata{bibtex}

\end{document}

% The format (A5) is selected to facilitate reading on small
% devices and should NOT be changed.
\documentclass[a4paper,10pt,oneside]{article}

% The package babel is loaded for Swdish with Swedish hyphenation,
% replaces "Contents" with "Innehållsförteckning, "References"
% with "Litteraturförteckning", etc.
\usepackage[swedish,english]{babel}

\usepackage[T1]{fontenc}

% The package "inputenc" lets us specify what character encoding
% has been used to save the .tex file. If your computer runs
% Linux, the encoding is probably "utf8" by default, while under
% Windows the default will probably be "latin1" The wrong
% character encoding may give strange signs instead of "å", "ä"
% and "ö" or may result in compilation errors.

%\usepackage[latin1]{inputenc} % Probably right if you use Windows
\usepackage[utf8]{inputenc}  % Probably right if you use Linux

% The packages listed below are optional and can be removed if you
% don't use them
\usepackage{graphicx}
\usepackage{cite}
\usepackage{url}
\usepackage{ifthen}
\usepackage{listings}
\usepackage{dirtree}
\usepackage{color}

% These two lines set up options for the listings package and
% can be removed if you don't use it, or changed if you, e.g,
% use another language than Java.
% For more information about the listings package see:
% ftp://ftp.tex.ac.uk/tex-archive/macros/latex/contrib/listings/listings.pdf
\definecolor{dkgreen}{rgb}{0,0.6,0}
\definecolor{gray}{rgb}{0.5,0.5,0.5}
\definecolor{mauve}{rgb}{0.58,0,0.82}
\def \lstlistingname {Example}
\lstset{frame=tb,
  language=Bash,
    aboveskip=3mm,
      belowskip=3mm,
        showstringspaces=false,
  columns=flexible,
    basicstyle={\footnotesize\ttfamily},
      numbers=none,
        numberstyle=\tiny\color{gray},
  keywordstyle=\color{blue},
    commentstyle=\color{dkgreen},
      stringstyle=\color{mauve},
        breaklines=true,
  breakatwhitespace=true,
    tabsize=3
    }
\usepackage{ifpdf}
\ifpdf
	\usepackage[hidelinks]{hyperref}
\else
	\usepackage{url}
\fi

% Change NR and TITLE below to appropriate values

\title{Lab 1 Report \\ Digital Forensics}

% Write the name and user namn for all participants in the group here.
% Separate persons with \and
\author{Anton Fluch \url{anfl4215} \and Johan Bäckström \url{jobc5829}}

\begin{document}

\maketitle
\newpage
% Here the actual report starts. Everything from here to the start of the
% bibilography should, of course, be removed before you start writing your
% own text.

\tableofcontents
\newpage

% INTRODUCTION
\section{Background}
The evidence for the case where provided in a .zip file named Lab1.zip. This file produced the following hash sums:
\\
\\ SHA256
\\ {\footnotesize 9c5d0bfbeccd75858426cfc84345e0a68687b0fc5662b715153aa88cefd60fba}
\\
\\
MD5\\
{\footnotesize c4a731672747131b8b457a77178ad386}
\\
\\
When opening the zip file the following folders and files where present:

\dirtree{%
.1 Lab1.
.2 Exercise1\textunderscore Hashing.
.3 erase.
.3 erase.exe.
.3 hello.
.3 hello (2).
.3 hello (3).
.3 hello (4).
.3 hello.exe.
.2 Exercise2\textunderscore File\textunderscore Identification.
.3 01.
.3 02.
.3 03.
.3 04.
.3 05.
.3 06.
.3 07.
.3 08.
.3 09.
.3 10.
.3 11.
.3 12.
.2 Exercise3\textunderscore Anti\textunderscore Files\textunderscore Forensics.
.3 c.mp3.
.3 Suspicious\textunderscore File.
.2 Exercise4\textunderscore Acquisition.
.3 winxp.dvi.
.2 Exercise5\textunderscore Cracking.
.3 casssh.pdf.
.3 ht.zip.tar.gpg.
.3 Untitled 1.ods.
.3 untitled.docx.
.3 untitled\textunderscore hash.txt.
.3 wallet1.dat.
.3 wallet2.
.2 Exercise6\textunderscore Steganography.
.3 c1l.png.
.3 c2l.png.
}
\newpage

%----------------------------------------------------
%-------------- EXERCISE 1
%----------------------------------------------------
\section{Exercise 1: Hashing}
In order to maintain the chain of custody and to uniquely identify all files, the hash sum for SHA256
\footnote{\url{https://en.wikipedia.org/wiki/SHA-2}} and MD5 \footnote{\url{https://en.wikipedia.org/wiki/MD5}}
where calculated for all the files in the folder Exercise1\textunderscore Hashing. In Kali Linux \footnote{\url{https://www.kali.org/}}
it is possible to calculate the hash sum of a file using the bash shell \footnote{\url{https://en.wikipedia.org/wiki/Bash_(Unix_shell)}}.
For example, if you type the command:
\begin{lstlisting}
sha256sum *
\end{lstlisting}
It will calculate and display the hash sum for the SHA256 algorithm for all the files in the folder you are currently standing.
This resulted in the following hash sums:

\begin{lstlisting}
sha256sum *
1c4ff4e490b15b2b214f26c5654decccbcbea9eb900f88649dc7b1e42341be56  erase
1316543942a8c6cd754855500cd37068edbbd8b31c4979d2825a4e799fed6102  erase.exe
fad878bd261840a4ea4a8277c546d4f46e79bbeb60b059cee41f8b50e28d0e88  hello
1316543942a8c6cd754855500cd37068edbbd8b31c4979d2825a4e799fed6102  hello (2)
60d13913155644883f130b85eb24d778314014c9479aedb5f6323bf38ad3a451  hello (3)
1c4ff4e490b15b2b214f26c5654decccbcbea9eb900f88649dc7b1e42341be56  hello (4)
60d13913155644883f130b85eb24d778314014c9479aedb5f6323bf38ad3a451  hello.exe

md5sum *
da5c61e1edc0f18337e46418e48c1290  erase
cdc47d670159eef60916ca03a9d4a007  erase.exe
da5c61e1edc0f18337e46418e48c1290  hello
cdc47d670159eef60916ca03a9d4a007  hello (2)
cdc47d670159eef60916ca03a9d4a007  hello (3)
da5c61e1edc0f18337e46418e48c1290  hello (4)
cdc47d670159eef60916ca03a9d4a007  hello.exe
\end{lstlisting}

An efficient way for matching hash sums is also possible using the same command, but we need to provide an option to it. Using the '-c' option
we can quickly check if a provided hash sum match with the file we are checking. First we need to create a new file with the hash sum for
all the files in the folder:

\begin{lstlisting}
sha256sum * > checksums.chk
\end{lstlisting}

This will create a new file named 'checksums.chk' which contains all the hash sums for the files in the folder. Then we run the command:

\begin{lstlisting}
sha256sum -c checksums.chk
\end{lstlisting}

The output should be the following:

\begin{lstlisting}
erase: OK
erase.exe: OK
hello: OK
hello (2): OK
hello (3): OK
hello (4): OK
hello.exe: OK
\end{lstlisting}

Which indicates that all the files currently stored in 'checksums.chk' match with all the files in the folder.
Now lets say that we have a specific file of interest which we know the hash sum of and we want to find out if the file is present on a computer.
This can be achieced by using the following command:

\begin{lstlisting}
find . -type f -exec sha256sum {} + | grep '^SHA256SUM'
\end{lstlisting}

{\footnotesize *Note that you need to replace 'SHA256SUM' with the actual hash value of the file}\\\\
This will search through the specified folder recursively for correlating SHA256 sums. If we run the command:

\begin{lstlisting}
find . -type f -exec sha256sum {} + | grep '^1c4ff4e490b15b2b214f26c5654decccbcbea9eb900f88649dc7b1e42341be56'
\end{lstlisting}

Which is the SHA256 sum of the file 'erase' mentioned above. We get the output:

\begin{lstlisting}
1c4ff4e490b15b2b214f26c5654decccbcbea9eb900f88649dc7b1e42341be56  ./erase
1c4ff4e490b15b2b214f26c5654decccbcbea9eb900f88649dc7b1e42341be56  ./hello (4)
\end{lstlisting}

This indicates that we found two files that both have the same SHA256 sum, 'erase' and 'hello (4)'.

This is a feature which should be considered as beneficial for a forensic examiner since it means that if you suspect that a file is present on a computer you can
easily find it. Even though the file name is changed the hash sums will be identical.

%----------------------------------------------------
%-------------- EXERCISE 1.2
%----------------------------------------------------
\subsection{Exercise 1.2: Comparison of Hashing Algorithms}
In this exercise the execution time of the SHA256 and the MD5 algorithm will be compared. The file that is used to compare the times can be found at \url{http://ipv4.download.thinkbroadband.com:8080/1GB.zip}
And should produce the following hash sums:

\begin{lstlisting}
sha256sum
5674e59283d95efe8c88770515a9bbc80cbb77cb67602389fd91def26d26aed2

md5sum
286e80b3b7420263038ab06d76774043
\end{lstlisting}

Using the 'stat' command we can get more information about the file:

\begin{lstlisting}
stat 1GB.zip
      File: 1GB.zip
      Size: 1073741824      Blocks: 2097160    IO Block: 4096   regular file
    Device: 801h/2049d      Inode: 13369385    Links: 1
    Access: (0664/-rw-rw-r--)  Uid: ( 1000/ fluchey)   Gid: ( 1000/ fluchey)
    Access: 2017-09-21 11:56:19.516000051 +0200
    Modify: 2017-09-21 11:55:49.996055229 +0200
    Change: 2017-09-21 11:55:50.100055012 +0200
     Birth: -
\end{lstlisting}

If we want to measure the time it takes to compute tha hash sums we can use the command 'time'.

\begin{lstlisting}
time sha256sum 1GB.zip
5674e59283d95efe8c88770515a9bbc80cbb77cb67602389fd91def26d26aed2  1GB.zip

real    0m6,065s
user    0m5,968s
sys     0m0,100s
\end{lstlisting}

\begin{lstlisting}
time md5sum 1GB.zip
286e80b3b7420263038ab06d76774043  1GB.zip

real    0m1,844s
user    0m1,732s
sys     0m0,108s
\end{lstlisting}

The 'time' command is described in more detail in the linux manual \footnote{http://man7.org/linux/man-pages/man7/time.7.html}.
\begin{itemize}
        \item 'real' The total time taken for the process to execute
        \item 'user' The amount of CPU time spent in user mode (Outside the kernel) within the process
        \item 'sys' The amount of CPU time spent in the kernel within the process
\end{itemize}

The SHA256 algorithm took a total of 6,065 seconds to run. The MD5 algorithm took a total of 1,844 seconds to run. This makes the MD5 algorithm 4,221 seconds faster.


%----------------------------------------------------
%-------------- EXERCISE 2
%----------------------------------------------------
\newpage
\section{Exercise 2: File Headers}
%----------------------------------------------------
%-------------- EXERCISE 3
%----------------------------------------------------
\newpage
\section{Exercise 3: Anti Files Forensics}
%----------------------------------------------------
%-------------- EXERCISE 4
%----------------------------------------------------
\newpage
\section{Exercise 4: Acquisition}
%----------------------------------------------------
%-------------- EXERCISE 5
%----------------------------------------------------
\newpage
\section{Exercise 5: }
%----------------------------------------------------
%-------------- EXERCISE 6
%----------------------------------------------------
\newpage
\section{Exercise 6 - Hashing}

%\bibliographystyle{plain}
%\bibliography{bibtex}
%\bibdata{bibtex}

\end{document}

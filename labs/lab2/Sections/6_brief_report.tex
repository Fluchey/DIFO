%Well done! Now that we have the internal document in place we need to make a brief report that the attorney and other non-technical colleagues can get read through swiftly. The brief report below, is an example of a forensics investigation report which contains the essentials of a forensics investigation. This brief version of the report contains all necessary information about the case and, needless to say, it maintains the chain of custody. The italic text is sample text from the source template. When writing your own report, italics should not be used for these sections below.

\section{Case 1 Brief Report}

%Provide the steps used to perform the investigation. This section will vary according to the type of investigation. Add sections as needed.

\begin{longtable}{p{0.5\textwidth}p{0.5\textwidth}}
\textbf{REPORT OF} & Requested Windows Forensic Investigation\\
&\\
\textbf{MEMORANDUM FOR} & \textit{The Super Secret Police} \\
			& \textit{Investigators Fluch \& Bäckström} \\
   & \textit{Super Secret police location}\\
&\\
\textbf{SUBJECT} & \textit{Forensic Media Analysis Report}\\
& \textit{SUBJECT: Gachev, Evgeny}\\
& Case Number: 1\\
\end{longtable}

\begin{enumerate}
\item \textbf{Status: Closed}\\

\item \textbf{Summary of Findings:}\\

GLÖM INTE FYLLA I SAMMANFATTNING AV VAD VI HITTAT	
	
\item \textbf{Items Analyzed}
\begin{longtable}{p{0.4\textwidth}p{0.6\textwidth}}
\textbf{\underline{TAG NUMBER}} & \textbf{\underline{ITEM DESCRIPTION}} \\
012345 & evgeny\textunderscore 40GB\textunderscore 2.raw
\end{longtable}

\item \textbf{Details of Findings}

\newcommand*{\MyIndent}{\hspace*{0.5cm}}%
\begin{tabular}{|l|}
	\hline
	\textbf{Summary of evidence file: (Found and verified in Encase)} \\ \hline
	\textbf{Image file name:} evgeny\textunderscore 40GB\textunderscore 2.raw \\
	\textbf{Image Name:} evgeny-laptop \\
	\textbf{Image file MD5:} e6dc38a4f42910729669990138f86265 \\
	\textbf{Image file sha1:} fc4fa4b0a97c5f080c48d103cb2dfa64734bb935 \\
	\textbf{No. hard disk partitions:} 1 \\
	\textbf{Partition information:} \\
	\MyIndent NTFS \\
	\MyIndent 8 sectors per cluster \\
	\MyIndent 512 bytes per sector = 4096 bytes per cluster \\
	\MyIndent Allocated space = 36690026496 bytes (34,2GB) \\
	\MyIndent Unallocated space = 8449069056 (7,9GB) \\
	\hline
	
	\textbf{Number of sectors on hard disk:} 88162304 \\
	\textbf{Sector size on hard disk:} 512 bytes \\
	\textbf{Disk size:} 45139095552 bytes (42GB) \\
	\hline
\end{tabular}

\begin{itemize}
	\newpage
	\item \textbf{System details}
	\begin{itemize}
		\item The {\footnotesize '\textbackslash Windows\textbackslash System32\textbackslash config\textbackslash software'} key contained information about the installed operating system which was
		\textbf{Windows 7 professional} with an install date of \textbf{Wednesday the 15th of July 2015, 10:35:41}

		\item The registry hives containing REGISTRY keys was copied from 
		\begin{itemize}
			\item \textbackslash Windows\textbackslash System32\textbackslash config\textbackslash system\\ 
			\item \textbackslash Windows\textbackslash System32\textbackslash config\textbackslash software\\
			\item \textbackslash Windows\textbackslash System32\textbackslash config\textbackslash security\\
			\item \textbackslash Windows\textbackslash System32\textbackslash config\textbackslash sam \\
			\item \textbackslash Windows\textbackslash System32\textbackslash config\textbackslash default \\
			\item \textbackslash Windows\textbackslash System32\textbackslash config\textbackslash system \\
			\item \textbackslash Users\textbackslash evgeny\textbackslash ntuser.dat \\
			Beside 'ntuser.dat' two log files named 'ntuser.dat.LOG1' and 'ntuser.dat.LOG2' were found and copied
			
			\item Withing the 'software' registry hive, under 
			Microsoft \textbackslash Windows \textbackslash CurrentVersion \textbackslash Run, the following information was found about programs that were configured to startup with the system.
			
			\begin{itemize}
				\item "C:\textbackslash Program Files\textbackslash Eraser.exe - atRestart".\\ 
				This is a program designed to remove files on startup.
			\end{itemize}
			
			\item The system is most probably a physical system.
			
		\end{itemize}
	\end{itemize}
	
	\newpage
	\item \textbf{User accounts}
	\begin{itemize}
		\item Looking through the folder\textbackslash Users\textbackslash within Encase, we found user accounts 
		
		\begin{itemize}
			\item 'Evgeny'
			\item 'Public'
			\item 'Default'
		\end{itemize} 
		
		\item Looking through the registry hive SAM with Registry Viewer, under
		\textbackslash SAM\textbackslash Domains\textbackslash Account\textbackslash Users\textbackslash Names\textbackslash
		we confirmed these user accounts.
		
		\begin{itemize}
			\item 'Evgeny'
			\item 'Guest'
			\item 'Administrator'
		\end{itemize}
		
		\item within the SAM registry hive, information was found about the login dates.\\ 
		Under\textbackslash SAM\textbackslash Domains\textbackslash Account\textbackslash Users\textbackslash Names\textbackslash evgeny, it was confirmed that the Hexadecimal value for the user (RID value) was 03E8. 
		Under\textbackslash SAM\textbackslash Domains\textbackslash Account\textbackslash Users\textbackslash 000003E8\textbackslash the following was found:
		
		\begin{itemize}
			\item User name: 'Evgeny'
			\item Last logon time: 11/6/2015 13:48:51 UTC
			\item Last password change time 7/15/2015 10:35:34 UTC.\\
			If Evgeny hasn't changed his password, this corresponds to the account creation time.
			\item Logon count: 33
			\item User password hint: "NoSuchThingBaby"
		\end{itemize}
\end{itemize}
	
	\item The installed programs that does not come with windows pre-installed that were found in:
	\textbackslash Program Files\textbackslash Program Files (x86) were:
	
	\begin{itemize}
		\item FBCIM (Facebook chat IM) 
		\item GNU-PG (GNU Privacy Guard) 
		\item Mozilla Firefox
		\item Mozilla Thunderbird
	\end{itemize}
	
	\newpage
	\item \textbf{Installed programs found}
	
	\begin{tabular}{|r|l|}
		\hline
		\textbf{7-zip} & Compression program \\
		\textbf{DVD Maker} & Used for creating DVD:s \\
		\textbf{Eraser} & Used for deletion of files \\
		\textbf{Facebook chat IM} & A client for facebook chat \\
		\textbf{Filezilla FTP Client} & A file transfer program \\
		\textbf{Gnu Privacy Guard (GNU-PG)} & Used for encrypting data \\
		\textbf{Hidden Tear} & A program used for creating ransomware \\
		\textbf{Mozilla Firefox} & An internet browser \\
		\textbf{Mozilla Thunderbird} & Email client \\
		\textbf{Tor Browser} & An internet browser used for anonymization \\
		\textbf{Truecrypt} & Used for encrypting data \\
		\textbf{Veracrypt} & A redefined version of truecrypt, also used for encrypting data \\
		\hline
	\end{tabular}
	
	\newpage
	\item \textbf{Program data found}
	\begin{itemize}
		\item \textbf{Thunderbird (Email client):} 
		\begin{itemize}
			\item email addresses found under "TO":
			\begin{itemize}
				\item evgeny.luckyone@mail.ru
			\end{itemize}
			\item email addresses found under "FROM":
			\begin{itemize}
				\item security@facebook.com
				\item welcome@corp.mail.ru
				\item mailapps@corp.mail.ru
				\item notification+zrdzvgcipoif@facebookmail.com
				\item info@luxurytravelmedia.com
				\item agent@corp.mail.ru
				\item MAILER-DAEMON@fallback7.mail.ru
				\item no-reply@ghostmail.com
				\item mailer-daemon@corp.mail.ru
				\item goranborisov@mailbox.org
				\item do-not-reply@keyserver2.pgp.com
				\item chingiz112@ghostmail.com
				\item kykypykv@ghostmail.org
			\end{itemize}
			
			\item Looking through the emails sent and received some interesting mails were found: 
			
			\begin{itemize}
				\item from: goranborisov@mailbox.org. The mail had an attached file named "encrypted.asc" of the size 4086 kB. The subject of the mail was "re:qwerty". 
				
				\item a lot of correspondence to and from Facebook, with the correspondence ending with a mail stating that:
				
				"You have requested a copy of your Facebook data"
				
				And after that:
				
				"Your Facebook download is ready"
				
			\end{itemize}
		\end{itemize}
	
	\item \textbf{FileZilla}
	\begin{itemize}
		
	\item Indication of attempted access to ftp was found with the local installation of FileZilla.
	
	\begin{itemize}
		\item "C:\textbackslash Users\textbackslash evgeny\textbackslash AppData\textbackslash Roaming\textbackslash FileZilla\textbackslash recenttservers.xml" contained the following connections:
		\item \textit{IP-Address:} 72.22.81.47:22 \textit{Username:} lucky456 \textit{Password hash:} bHVja3k0NTY=
		\item \textit{IP-Address:} 31.220.43.103:999 \textit{Username:} lucky456 \textit{Password hash:} cXZhYzE1MTBTWlpKcDlj
		\item \textit{IP-Address:} 31.220.43.103:22 \textit{Username:} lucky456 \textit{Password hash:} cXZhYzE1MTBTWlpKcDlj\\
		
	\end{itemize}
	\end{itemize}
	\item \textbf{Gnupg}
		\begin{itemize}
			\item under C: \textbackslash Users \textbackslash evgeny \textbackslash AppData \textbackslash Local \textbackslash Temp \textbackslash gpg-hwyPbw \textbackslash tempin.txt - the following file was found concerning a connection to the GnuPG key server
			\begin{lstlisting}
				# This is a GnuPG 2.0.28 keyserver communications file
				VERSION 1
				PROGRAM 2.0.28
				SCHEME hkp
				HOST keyserver2.pgp.com
				PORT 11371
				PATH /
				COMMAND GET
				
				0x424A6212

			\end{lstlisting}
		\end{itemize}

		\end{itemize}
	
	\newpage
	\item Virtual operating systems found
	
	\begin{tabular}{|r|l|}
		\hline
		\textbf{Kali Linux} & Debian-derived Linux distribution designed for digital forensics and penetration testing \\
		\textbf{Tails Linux} & Security-focused Debian-based Linux distribution aimed at preserving privacy and anonymity \\
		\hline
	\end{tabular}

	\newpage
	\item Network data found
	
	\begin{itemize}
	
	\item The registry key "\textbackslash system\textbackslash ControlSet001\textbackslash Services\textbackslash Tcpip\textbackslash Parameters\textbackslash" contained the following information
	\begin{itemize}
		\item \textit{Name:} ICSDomain  \textit{data: mshome.net}
		\item \textit{Name:} Hostname  \textit{data: evgeny-laptop}
		
	\end{itemize}
	
	\item The registry key \textbackslash Interfaces\textbackslash \{AB3E992B-CC1C-464D-A1D6-9EB82846715C\} contained
	
	\begin{itemize}
		\item \textit{Name:} DhcpIPAddress  \textit{data: 172.17.1.183}
		\item \textit{Name:} DhcpNameServer  \textit{cs2lab.dsv.su.se}
	\end{itemize}
	
\end{itemize}	
	
	\newpage
	\item The investigation has so far found a lot of information within the windows registry, as stated above. The registry hives searched through has been:
	
	\item 
	\begin{itemize}
		\item \textbackslash Windows\textbackslash System32\textbackslash config\textbackslash system\\ 
		\item \textbackslash Windows\textbackslash System32\textbackslash config\textbackslash software\\
		\item \textbackslash Windows\textbackslash System32\textbackslash config\textbackslash security\\
		\item \textbackslash Windows\textbackslash System32\textbackslash config\textbackslash sam \\
		\item \textbackslash Windows\textbackslash System32\textbackslash config\textbackslash default \\
		\item \textbackslash Windows\textbackslash System32\textbackslash config\textbackslash system \\
	\end{itemize} 
	
	\item Withing the software registry hive, information can be found about installed and uninstalled programs. Within \textbackslash Software \textbackslash Microsoft \textbackslash Windows \textbackslash CurrentVersion \ Installer \textbackslash Folders - the following entries was found to be interesting.
	
	\begin{itemize} 
		\item Name: C:\textbackslash Program Files \textbackslash Oracle \textbackslash VirtualBox \textbackslash
		\item Name: C:\textbackslash Program Files \textbackslash Eraser \textbackslash
		\item Name: C:\textbackslash Program Files (x86) \textbackslash Java \textbackslash\\
	\end{itemize}
	
	\item  Within \textbackslash Software \textbackslash Microsoft \textbackslash Windows \textbackslash CurrentVersion \ Uninstall - the following entries was found to be interesting.
	
	\begin{itemize}
		\item \textbraceleft 66AB13EA-E7D2-4CFC-9B66-8E9EE44C89EE \textbraceright \\
		Name: Comments: Data: Secure Data Removal For Windows \\
		Name: DisplayName Data: Eraser 6.2.0.2969 \\
		Name: InstallDate Data: 2015-08-17 \\
		
		\item \textbraceleft FCD0B365-2189-45F3-9AF2-2BCED86C121A \textbraceright \\
		Name: DisplayName Data: Oracle VM VirtualBox 5.0.0 \\
		Name: InstallDate Data: 2015-07-24 \\
		
		
	\end{itemize}
	
	
\end{itemize}

\end{enumerate}

\noindent \textit{IMA D. EXAMINER}	\hfill Released by {\wesa Mr. X}\\
\textit{N.N} Computer Forensic Examiner
\clearpage
